\documentclass[14pt]{beamer}
	
\input{../../template/slideTemplate.txt}
\renewcommand{\DEBUG}{0}
\renewcommand{\PRESENTATION}{0}

\subtitle{Sim, você já usou}
\usecolortheme[named=slideOrange]{structure}

\begin{document}
	
	\begin{frame}
		\titlepage
	\end{frame}

	\begin{frame}
		\tableofcontents
	\end{frame}

	\section{Onde usamos}
		\begin{frame}{Usando aqui}
			\begin{itemize}
				\presentationPause\item Você se lembra dos irmãos \textit{console}? \presentationPause \basicCode{cin} e \basicCode{cout}
				\presentationPause\item Já se perguntou como os operadores de deslocamento fazem isso? \presentationPause \basicCode{>>,<<}
				\presentationPause\item Lembra do desafio de somar $\overrightarrow{\mathsf{v}}$?
				\presentationPause\item Imagina poder fazer algo como \basicCode{v1 + v2} com uma \basicCode{struct}
				\presentationPause\item Tudo isso é possíve com sobrecarregarga de operadores
			\end{itemize}
			\presentationPause\lstinputlisting[numbers=none]{../../code/minimal/basic.example.HW.cin.cpp}
			% \hline
			\presentationPause\lstinputlisting[numbers=none]{../../code/minimal/basic.example.HW.cout.cpp}
		\end{frame}

		\begin{frame}{E ainda\dots}
			\presentationPause\lstinputlisting[numbers=none]{../../code/class/overload.console.cpp}
		\end{frame}

	\section{Sobrecarregando}
		\begin{frame}
			\begin{itemize}
				\presentationPause\item Consiste em alterar a operação do operador
				\presentationPause\item A sintaxe de sobrecarga é padrão
			\end{itemize}
			\presentationPause\lstinputlisting[numbers=none]{../../code/class/overload.sintax.cpp}
			\begin{itemize}
				\presentationPause\item Chatices:
				\begin{itemize}
					\presentationPause\item Nem todos os operadores são sobrecarregáveis
					\presentationPause\item Não pode ser apenas em tipos primitivos
					\presentationPause\item Ordem de precedência não pode ser alterada
					\presentationPause\item Ordem de agrupamento não pode ser alterada \presentationPause \textit{É o que?}
					\presentationPause\item Novos operadores não podem ser criados
				\end{itemize}
			\end{itemize}
		\end{frame}

		\begin{frame}{Inteiro}
			\presentationPause
			\begin{columns}
				\begin{column}{.5\textwidth}
					\lstinputlisting[linerange={1-18}]{../../code/class/overload.example.1.cpp}
				\end{column}
				\begin{column}{.4\textwidth}
					\lstinputlisting[firstnumber=last,linerange={20}]{../../code/class/overload.example.1.cpp}
				\end{column}
			\end{columns}
		\end{frame}

	\section{Metacarregando}
		\begin{frame}{Retornos estrambólicos}
			\begin{itemize}
				\presentationPause\item \textit{Ah, mas eu lembro que os \textbf{consoles} aceitavam mais de um\dots}
				\presentationPause\item \textit{E eu testei aqui e deu errado...}
				\presentationPause\item \textit{Tem uma forma de fazer isso?}
				\presentationPause\item Sim!
				\presentationPause\item Você pode retornar o próprio objeto!
				\presentationPause\item \textit{É o que?}
			\end{itemize}
		\end{frame}

		\begin{frame}{\textit{Mind blow}}
			\begin{itemize}
				\presentationPause\item Podemos retornar variáveis de funções
				\presentationPause\item Também podemos retornar tipos abstratos com funções
				\presentationPause\item E podemos retornar objetos com funções do tipo de sua classe
				\presentationPause\item O código anterior pode ficar mais interessante\dots
			\end{itemize}
		\end{frame}
		
		\begin{frame}{Interessantemente}
			\presentationPause
			\begin{columns}
				\begin{column}{.5\textwidth}
					\lstinputlisting[linerange={1-18}]{../../code/class/overload.example.2.cpp}
				\end{column}
				\begin{column}{.4\textwidth}
					\lstinputlisting[firstnumber=last,linerange={20}]{../../code/class/overload.example.2.cpp}
				\end{column}
			\end{columns}
		\end{frame}

		
	\section{Este ponteiro}
		\begin{frame}{Mas\dots}
			\begin{itemize}
				\presentationPause\item Ficou meio ambíguo\dots
			\end{itemize}
			\presentationPause\lstinputlisting[numbers=none,linerange={14-17}]{../../code/class/overload.example.1.cpp}
			\presentationPause\lstinputlisting[numbers=none,linerange={14-17}]{../../code/class/overload.example.2.cpp}
			\begin{itemize}
				\presentationPause\item Qual \basicCode{getValue()} é de quem?
			\end{itemize}
		\end{frame}

		\begin{frame}{Este}
			\begin{itemize}
				\presentationPause\item Para evitar isso existe um ponteiro especial
				\presentationPause\item Além de evitar isso, pode fazer mais coisas
				\presentationPause\item Este ponteiro aponta para o objeto em questão \presentationPause\textit{Em questão?}
				\presentationPause\item É denotado pela paravra-chave \basicCode{this}
			\end{itemize}
			\presentationPause\lstinputlisting[numbers=none,linerange={2-7}]{../../code/class/this.sintax.cpp}
		\end{frame}

		\begin{frame}{Novamente\dots}
			\presentationPause\lstinputlisting[linerange={1-18}]{../../code/class/overload.example.3.cpp}
		\end{frame}
	

	\section{Hora de brincar}
		\begin{frame}
			\begin{center}\Huge
				Vamos testar!
			\end{center}
		\end{frame}
	
\end{document}