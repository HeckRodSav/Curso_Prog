\documentclass[14pt]{beamer}
	
\input{../../template/slideTemplate.txt}
\renewcommand{\DEBUG}{0}
\renewcommand{\PRESENTATION}{0}

\subtitle{Vire o papel}
\usecolortheme[named=slideGreen]{structure}

\begin{document}
	
	\begin{frame}
		\titlepage
	\end{frame}

	\begin{frame}
		\tableofcontents
	\end{frame}

	\section{Polimorfismo}
		\begin{frame}{Qual nome?}
			\begin{itemize}
				\presentationPause\item Fazer função para calcular média
				\presentationPause\item Média de 2 valores\presentationPause, 3 valores\presentationPause, 4 valores\presentationPause\dots
				\presentationPause\item Qual nome utilizar?
				\presentationPause\lstinputlisting[numbers=none, linerange={1-14}]{../../code/functions/poly.non.cpp}
				\presentationPause\item E se você pudesse utilizar o mesmo nome?
			\end{itemize}
		\end{frame}

		\begin{frame}{O mesmo nome}
			\begin{itemize}
				\presentationPause\item Você pode!
				\presentationPause\item \textit{E se ligar agora, tem o incrível desconto de\dots}
				\presentationPause\item Como?
				\presentationPause\item Simples, faça!
				\presentationPause\lstinputlisting[numbers=none, linerange={1-14}]{../../code/functions/poly.average.cpp}
				\presentationPause\item O compilador resolve (argumentos)
			\end{itemize}
		\end{frame}

	\section{Recursão}
		\begin{frame}{Explicando recursão}
			\begin{itemize}
				\presentationPause\item Como explicar recursão?
				\presentationPause\item Explicar fazendo recursão
				\presentationPause\item Mas o que é recursão?
				\presentationPause\item Fazer recursão sem saber recursão
				\presentationPause\item Como explicar recursão?
				\presentationPause\item Explicar fazendo recursão
				\presentationPause\item Mas o que é recursão?
				\presentationPause\item Fazer recursão sem saber recursão
				\presentationPause\item Como explicar recursão?
				\presentationPause\item Explicar fazendo recursão
				\presentationPause\item Mas o que é recursão?
				\presentationPause\item Fazer recursão sem saber recursão
				\presentationPause\item \dots
			\end{itemize}
		\end{frame}

		\begin{frame}{É o que?}
			\begin{itemize}
				\presentationPause\item Um procedimento que chama a sí próprio
				\presentationPause\item Precisa de uma limitação
				\presentationPause\item \textit{stack overflow}
				\presentationPause\item Simplificar problema
				\presentationPause\item Condição de parada
				\presentationPause\item Passo de recursão
				\presentationPause\item Caso trivial
			\end{itemize}
		\end{frame}

		\begin{frame}{Você já viu um}
			\begin{itemize}
				\presentationPause\item A função recursiva mais famosa é o fatorial!
				\presentationPause\item O que é o fatorial?
				
					\presentationPause\begin{equation}\label{equation.fatorial.1}
						f(n) = n! \equiv n\cdot(n-1)\cdot(n-2)\cdot {\dots} \cdot2\cdot1
					\end{equation}

					\presentationPause\begin{equation}\label{equation.fatorial.2}
						n! = \left\{
						\begin{array}{lcl}
							n\leq1&\Rightarrow&1\\
							n\nleq1&\Rightarrow&n\cdot(n-1)!\\
						\end{array}\right.
					\end{equation}

					\presentationPause\begin{equation}\label{equation.fatorial.3}
						f(n) = n! = n\cdot(n-1)! = n\cdot f(n-1)
					\end{equation}
				\presentationPause\item E como fica um código disso?
				\presentationPause\lstinputlisting[numbers=none]{../../code/functions/recursion.cpp}
			\end{itemize}
		\end{frame}

	\section{Hora de brincar}
		\begin{frame}
			\begin{center}\Huge
				Vamos testar!
			\end{center}
		\end{frame}

	
\end{document}
