\documentclass[11pt]{beamer}
	
	\input{../../template/slideTemplate.txt}
\renewcommand{\DEBUG}{0}
\renewcommand{\PRESENTATION}{1}

\subtitle{Tipo assim\dots}
\usecolortheme[named=slideBlue]{structure}

\begin{document}

\begin{frame}
	\titlepage
\end{frame}

\begin{frame}
	\tableofcontents
\end{frame}

\section{Tipos de dados}
	\begin{frame}{Tipos}
		\begin{itemize}
			\presentationPause\item Todo programa precisa armazenar dados, até o código mínimo utiliza uma \presentationPause\item quantidade de memória.
			\presentationPause\item Podem ser armazenados dados numéricos, de texto, estados lógicos.
			\presentationPause\item Os tipos aqui apresentados são denominados \emph{tipos primitivos}.
		\end{itemize}
	\end{frame}\begin{frame}{Tipos}
		\begin{itemize}
			\presentationPause\item Estados lógicos são armazenados nas memórias tipo \basicCode{bool}
			\presentationPause\item Caracteres de texto são armazenados nas memórias tipo \basicCode{char}
			\presentationPause\item Valore numéricos apresentam dois tipos de armazenamento:
			\begin{itemize}
				\presentationPause\item Números inteiros nos tipo \basicCode{int}
				\presentationPause\item Números flutuantes nos tipo \basicCode{float} e \basicCode{double}
			\end{itemize}
		\end{itemize}
		\presentationPause Há ainda um tipo especial, que não armazena dados, o tipo \basicCode{void}, seu será explicado daqui algumas aulas.
	\end{frame}

	\begin{frame}{Modificadores}
		\begin{itemize}
			\presentationPause\item Os modificadores de faixa são palavras-chave que alteram os valores registráveis por um tipo.
			\presentationPause\item Os modificadores \basicCode{signed} e \basicCode{unsigned} alteram a signatura da faixa.
			\presentationPause\item Os modificadores \basicCode{short} e \basicCode{long} alteram o comprimento da faixa.
		\end{itemize}
	\end{frame}\begin{frame}{Modificadores}
		\begin{table}[h]
			\footnotesize\centering
			\caption{Relação de faixa e tamanhos de memória para tipos primitivos com modificadores de faixa}
			\label{table.sign.range}
			\begin{tabular}{rccc}
				\multicolumn{1}{c}{código}	& tamanho (B) & valor mínimo & valor máximo \\\hline
		
				\presentationPause\basicCode{bool}								& 1  & 0 & 1 \\\hline
		
				\presentationPause\basicCode{signed char}					& 1  & -127 & 126 \\
				\basicCode{char}								& 1  & -127 & 126 \\
				\basicCode{unsigned char}				& 1  & 0 & 255\\\hline
		
				\presentationPause\basicCode{signed short int}		& 2  & -32768 & 32767 \\
				\basicCode{short int}						& 2  & -32768 & 32767 \\
				\basicCode{unsigned short int}	& 2  & 0 & 64535\\\hline
		
				\presentationPause\basicCode{signed int}					& 4  & -2147483648 & 2147483647 \\
				\basicCode{int}									& 4  & -2147483648 & 2147483647 \\
				\basicCode{unsigned int}				& 4  & 0 & 4294967295\\\hline
		
				\presentationPause\basicCode{signed long int}			& 8  & -9223372036854775808 & 9223372036854775807 \\
				\basicCode{long int}						& 8  & -9223372036854775808 & 9223372036854775807 \\
				\basicCode{unsigned long int}		& 8  & 0 & 18446744073709551616\\\hline
		
				\presentationPause\basicCode{float}								& 4  & $1.2\cdot10^{-38}$ & $3.4\cdot10^{+38}$\\\hline
		
				\presentationPause\basicCode{double}							& 8  & $1.73\cdot10^{-308}$ & $1.7\cdot10^{+308}$ \\
				\basicCode{long double}					& 16 & $3.4\cdot10^{-4932}$ & $3.4\cdot10^{+4932}$ 
			\end{tabular}
		\end{table}
	\end{frame}

\section{Variáveis}
	\begin{frame}{Variáveis}
		De nada adianta \emph{existir} um tipo de armazenamento de dados se não soubermos usá-lo.

		\presentationPause Uma variável é declarada colocando a lista de modificadores, o tipo e o nome da variável.

		\presentationPause\lstinputlisting[numbers=none,linerange={2-2}]{../../code/variables/declaration.cpp}

		\presentationPause Quando uma variável é declarada, ela pode vir com lixo de memória, para evitar isso, declaramos a variável com um valor de inicialização, seguindo a sintaxe:

		\presentationPause\lstinputlisting[numbers=none,linerange={2-2}]{../../code/variables/inicialization.cpp}
	\end{frame}\begin{frame}{Variáveis}
		\lstinputlisting[linerange={2-12}]{../../code/variables/examples.cpp}
		\presentationPause\emph{Note os detalhes!}
	\end{frame}

\section{Exibindo}
	\begin{frame}{\textit{printf}}
		\begin{itemize}
			\presentationPause\item O \basicCode{prinft} é uma das opções para exibição na tela.
			\presentationPause\item Pular linha? Caracter especial! <\href{http://en.cppreference.com/w/cpp/language/escape}{en.cppreference.com/w/cpp/language/escape}>
			\presentationPause\item Exibir variáveis? Sequência especial! <\href{http://en.cppreference.com/w/cpp/io/c/fprintf}{en.cppreference.com/w/cpp/io/c/fprintf}>
		\end{itemize}
		\presentationPause\lstinputlisting[linerange={2-2}, numbers=none]{../../code/functions/printf.sintax.cpp}
	\end{frame}\begin{frame}{\textit{printf}}
		O código:
		\lstinputlisting[linerange={2-5}, numbers=none]{../../code/functions/printf.example.cpp}

		\presentationPause
		Gera a saída:
		\lstinputlisting[language={}, numbers=none]{../../code/functions/printf.out.txt}
	\end{frame}

\section{Alterações}
	\begin{frame}{Modificando variáveis}
		\begin{itemize}
			\presentationPause\item Variáveis que não variam não são variáveis
			\presentationPause\item As alterações nas variáveis dependem de seu tipo
			\presentationPause\item Existem três grupos de operadores: unários, binários e ternários:
			\begin{itemize}
				\presentationPause\item[Unário] utiliza apenas um valor
				\presentationPause\item[Binário] utiliza dois valores
				\presentationPause\item[Ternário] utiliza três valores
			\end{itemize}
			\presentationPause\item Todo operador retorna o valor de sua operação
		\end{itemize}
	\end{frame}

	\begin{frame}{Sinalizadores}
		\begin{itemize}
			\presentationPause\item Equivalem aos sinais matemáticos
			\presentationPause\item Fazem o mesmo que multiplicar um número por $+1$ ou $-1$
			\presentationPause\item Não alteram o valor registrado
		\end{itemize}
		\presentationPause\lstinputlisting[numbers=none,linerange={2-4}]{../../code/operators/sign.sintax.cpp}
	\end{frame}\begin{frame}{Sinalizadores}
		\lstinputlisting{../../code/operators/sign.example.cpp}
	\end{frame}

	\begin{frame}{Atribuidor simples}
		\begin{itemize}
			\presentationPause\item O operador mais utilizado é o atribuidor simples
			\presentationPause\item Ele atribui o valor a direita à variável a direita
			\presentationPause\item Cuidado para não inverter
			\presentationPause\item Não atribuir tipos a variáveis de outros tipos
		\end{itemize}
		\presentationPause\lstinputlisting[numbers=none,linerange={2-2}]{../../code/operators/assignment.sintax.cpp}
	\end{frame}\begin{frame}{Atribuidor simples}
		\lstinputlisting{../../code/operators/assignment.example.cpp}
	\end{frame}

	\begin{frame}{Aritméticos}
		\begin{itemize}
			\presentationPause\item As quatro operações básicas da matemática
			\presentationPause\item Os símbolos padrão
			\presentationPause\item Não alteram o valor registrado
			\presentationPause\item Cuidado para não dividir por 0
		\end{itemize}
		\presentationPause\lstinputlisting[numbers=none,linerange={2-6}]
		{../../code/operators/arithmetic.sintax.cpp}
		\presentationPause\begin{equation}\nonumber
			\begin{array}{cc}
				\text{\basicCode{E}} & \multicolumn{1}{|c}{\text{\basicCode{F}}} \\ \cline{2-2} 
				\text{\basicCode{G}} & \text{\basicCode{H}}
			\end{array} \Rightarrow \begin{array}{cc}
				13 & \multicolumn{1}{|c}{5} \\ \cline{2-2} 
					3 & 2		
			\end{array}
		\end{equation}
	\end{frame}\begin{frame}{Aritméticos}
		\lstinputlisting{../../code/operators/arithmetic.example.cpp}
	\end{frame}

\begin{frame}{Comparadores}
	\begin{itemize}
		\presentationPause\item Verificar se valores são iguais ou diferentes
		\presentationPause\item Descobrir se um valor é maior que outro
	\end{itemize}
	\presentationPause\lstinputlisting[numbers=none,linerange={2-7}]{../../code/operators/comparison.sintax.cpp}
\end{frame}\begin{frame}{Comparadores}
	\lstinputlisting{../../code/operators/comparison.example.cpp}
\end{frame}

\section{Hora de brincar}
	\begin{frame}
		\begin{center}\Huge
			Vamos testar!
		\end{center}
	\end{frame}

\end{document}
	