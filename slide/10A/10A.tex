\documentclass[14pt]{beamer}
	
\input{../../template/slideTemplate.txt}
\renewcommand{\DEBUG}{0}
\renewcommand{\PRESENTATION}{0}

\subtitle{Eu sou seu pai}
\usecolortheme[named=slideRed]{structure}

\begin{document}
	
	\begin{frame}
		\titlepage
	\end{frame}

	\begin{frame}
		\tableofcontents
	\end{frame}

	\section{Olhos da mãe, nariz do pai}
		\begin{frame}{Herdando características}
			\begin{itemize}
				\presentationPause\item O que temos em comum com chimpanzés?
				\presentationPause\item Polegares opositores\presentationPause, quatro membros\presentationPause, globos oculares frontais
				\presentationPause\item O que temos de diferença com chimpanzés?
				\presentationPause\item Menos pelos\presentationPause, mais massa cinzenta\presentationPause, sistema de comunicação superior
				\presentationPause\item Talvez tenhamos algum antepassado comum
				\presentationPause\item Então podemos generalizar estas características
				\presentationPause\item Pensando em código\presentationPause, só escrever um bloco uma vez
				\presentationPause\item Neste caso, descrever polegares opositores apenas uma vez
			\end{itemize}
		\end{frame}

	\section{Hora de brincar}
		\begin{frame}
			\begin{center}\Huge
				Vamos testar!
			\end{center}
		\end{frame}
	
\end{document}