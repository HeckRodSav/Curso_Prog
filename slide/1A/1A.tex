\documentclass[11pt]{beamer}

\newcommand{\DEBUG}{0}
\newcommand{\PRESENTATION}{0}
\input{../slideTemplate.txt}

\subtitle{Introducto}
\usecolortheme[named=slideBlue]{structure}

\begin{document}

\titlepage

\begin{frame}
	\tableofcontents
\end{frame}

\section{Sobre o curso}
	\begin{frame}{Sobre o professor}
		\begin{itemize}
			\presentationPause\item Técnico de mecatrônica - ETEC Jorge Street - 2014
			\presentationPause\item Ingressante de 2015
			\presentationPause\item Programador na equipe UFABC Rocket Design
			\presentationPause\item Ciência da Computação
			\presentationPause\item Engenharia de Informação
			\presentationPause\item Entusiasta de tecnologia
		\end{itemize}
	\end{frame}	


	\begin{frame}{Sobre o curso}
		\begin{itemize}
			\presentationPause\item Essencialmente presencial
			\presentationPause\item Temos um site! <\href{https://sites.google.com/view/heckrodsav}{sites.google.com/view/heckrodsav}>
			\presentationPause\item Presença: chamada toda aula
			\presentationPause\item Avaliação: pequenas atividades ao longo do curso
			\presentationPause\item Aulas em horários {\itshape relativamente} flexíveis
			\presentationPause\item Temos um servidor! <\href{https://c9.io/heck0cpp0course}{c9.io/heck0cpp0course}>
			\presentationPause\item Esquema de cores para indicar nível de dificuldade
		\end{itemize}
		\begin{center}
			\palette
		\end{center}
	\end{frame}	

\section{Um pouco sobre C++}
	\begin{frame}{Breve história}
		\begin{itemize}
			\presentationPause\item Surgiu em 1979
			\presentationPause\item Deriva diretamente da linguagem C
			\presentationPause\item Primeira padronização em 1998, depois 2003, 2011 e 2014
			\presentationPause\item Próxima atualização em 2017
			\presentationPause\item Multiparadigma!
		\end{itemize}
	\end{frame}	

	\begin{frame}{Porque usar C++?}
		\begin{itemize}
			\presentationPause\item Realmente rápido, perdendo para linguagens de máquina, como Assembly
			\presentationPause\item Amplamente difundida, utilizada e referenciada
			\presentationPause\item Acesso a hardware
			\presentationPause\item Respeita o Programador
			\presentationPause\item Falar mais vantagens é \textit{spoiler}
		\end{itemize}
	\end{frame}

	\begin{frame}{O que é feito em C++?}
		\begin{multicols}{2}
			\begin{itemize}
				\presentationPause\item Adobe Photoshop
				\presentationPause\item Arduino
				\presentationPause\item Counter-Strike
				\presentationPause\item GTA
				\presentationPause\item Half-Life
				\presentationPause\item JVM - Java
				\presentationPause\item Linux
				\presentationPause\item Maya
				\presentationPause\item Microsoft Office
				\presentationPause\item Mozilla Firefox
				\presentationPause\item The Sims
				\presentationPause\item Tibia
				\presentationPause\item Unreal Engine 4
				\presentationPause\item Windows
			\end{itemize}		
		\end{multicols}
	\end{frame}
	
	\section{Um pouco de código}
	\begin{frame}{O código mínimo}
		O código mínimo é base pra entendimento da linguagem, porém sua compreensão requer uma base sólida de conceitos que serão abordados posteriormente.

		Ele faz absolutamente nada. :)
		\lstinputlisting{../../code/minimal.cpp}
	\end{frame}
	
	\begin{frame}{Sintaxe básica}
		\presentationPause\begin{center}
			\noindent\lstinline[basicstyle={\Huge\ttfamily}]k\{;\}k
		\end{center}
		\presentationPause\begin{center}
			\noindent\lstinline[basicstyle={\Huge\ttfamily},mathescape]|A$\neq$a|
		\end{center}
		\presentationPause\begin{center}
			\noindent\lstinline[basicstyle={\Huge\ttfamily},commentstyle={}]|//Linear|
		\end{center}
		\presentationPause\begin{center}
			\noindent\lstinline[basicstyle={\Huge\ttfamily},commentstyle={}]|/*Blocular*/|
		\end{center}
	\end{frame}

\section{GNU/Linux e o GCC}
	\begin{frame}{Onde programar?}
		\begin{itemize}
			\presentationPause\item Code::Blocks
			\presentationPause\item O GNU/Linux
			\presentationPause\item \textit{Open source}
			\presentationPause\item O GCC
		\end{itemize}
		\presentationPause\lstinputlisting[language=bash, numbers=none]{../../code/bash.compile.sh}
	\end{frame}

	\begin{frame}{Hello Wolrd!}
		O programa mais comum para iniciar é o \textit{Hello Wolrd}, que consiste em uma saída de texto simples.
		O C++ permite sua escrita de várias formas diferentes.

		\only<1>{\lstinputlisting{../../code/minimal.example.HW.puts.cpp}
		É a forma mais arcaica de apresentar uma informação, a mais complicada.}

		\only<2>{\lstinputlisting{../../code/minimal.example.HW.printf.cpp}
		É a forma utilizada no C, que permite formatação de dados de maneira mais intuitiva.}

		\only<3>{\lstinputlisting{../../code/minimal.example.HW.cout.cpp}
		É a forma moderna do C++, utiliza o conceito de transmissão de dados, a mais simples.}
	\end{frame}

\section{Hora de brincar}
	\begin{frame}
		\begin{center}\Huge
			Vamos testar!
		\end{center}
	\end{frame}

\end{document}
