\documentclass[14pt]{beamer}
	
\input{../../template/slideTemplate.txt}
\renewcommand{\DEBUG}{0}
\renewcommand{\PRESENTATION}{0}

\subtitle{Truncando}
\usecolortheme[named=slideTurquoise]{structure}

\begin{document}
	
	\begin{frame}
		\titlepage
	\end{frame}

	\begin{frame}
		\tableofcontents
	\end{frame}

	\section{Modelando valores}
	\begin{frame}{Problemas}
		\begin{itemize}
			\presentationPause\item Preciso utilizar um operador não definido para um tipo
			\presentationPause\item \basicCode{\%} para \basicCode{float} ou \basicCode{double}
			\presentationPause\item Preciso arredondar um número conforme uma regra
			\presentationPause\item Não posso truncar um valor
			\presentationPause\item Fazer contas de \basicCode{int} com \basicCode{bool} no meio
			\presentationPause\item Verificar se uma letra é maiúscula ou minúscula \presentationPause(mentira)
			\presentationPause\item \textit{hashing} em criptografia
		\end{itemize}
	\end{frame}

	\begin{frame}{Solução}
		Casting!
		\begin{itemize}
			\presentationPause\item Mais um operador
			\presentationPause\item Não possui um símbolo \presentationPause\frownie
			\presentationPause\item Converte tipos em tipos
			\presentationPause\item Não altera valor salvo
			\presentationPause\item Não muda o tipo da variável, só retorna o valor convertido
			\presentationPause\item Lembra uma função
		\end{itemize}
		\presentationPause\lstinputlisting[numbers=none, linerange={2-3}]{../../code/operators/casting.sintax.cpp}
	\end{frame}
	
	\begin{frame}{Aplicação}
		\presentationPause\lstinputlisting[numbers=none]{../../code/operators/casting.example.cpp}
	\end{frame}


	\section{Hora de brincar}
		\begin{frame}
			\begin{center}\Huge
				Vamos testar!
			\end{center}
		\end{frame}

\end{document}