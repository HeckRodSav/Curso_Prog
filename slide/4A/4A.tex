\documentclass[14pt]{beamer}
	
\input{../../template/slideTemplate.txt}
\renewcommand{\DEBUG}{0}
\renewcommand{\PRESENTATION}{0}

\subtitle{Sequencial}
\usecolortheme[named=slideTurquoise]{structure}

\begin{document}
	
	\begin{frame}
		\titlepage
	\end{frame}

	\begin{frame}
		\tableofcontents
	\end{frame}


	\section{Várias variáveis}
		\begin{frame}{Começando de outra forma}
			\begin{itemize}
				\presentationPause\item Vamos começar com $n=5$
				\presentationPause\item Faça um programa que lê $n$ valores
				\presentationPause\item Exiba os $n$ valores na tela
				\presentationPause\item Calcule a média dos $n$ valores
				\begin{columns}
					\begin{column}{.15\textwidth}
						\presentationPause\lstinputlisting[numbers=none]{../../code/array/N.declaration.cpp}
					\end{column}
					\begin{column}{.20\textwidth}
						\presentationPause\lstinputlisting[numbers=none]{../../code/array/N.read.cpp}
					\end{column}
					\begin{column}{.30\textwidth}
						\presentationPause\lstinputlisting[numbers=none]{../../code/array/N.average.cpp}
					\end{column}
					\begin{column}{.35\textwidth}
						\presentationPause\lstinputlisting[numbers=none]{../../code/array/N.show.cpp}
					\end{column}
				\end{columns}
				\presentationPause\item Agora com $n=6$, \presentationPause$n=7$, \presentationPause$n=10$, \presentationPause$n=100$\dots
				\presentationPause\item Existe uma forma de generalizar isso\dots
			\end{itemize}
		\end{frame}

		\begin{frame}{Motivacional}
			\presentationPause\lstinputlisting[caption={Declarando dezenas de variáveis}, label=code.array.vararray]{../../code/array/vararray.cpp}
		\end{frame}

	\section{Vetores}
		\begin{frame}{Sequência padrão}
			\begin{itemize}
				\presentationPause\item A forma que utilizamos é chamada \emph{vetor}
				\presentationPause\item Conjuntos de números sequenciados em relação a uma base
				\presentationPause\item Para nós, variáveis sequenciadas
				\presentationPause\item Acessíveis por um índice
				\presentationPause\item Um operador novo! \presentationPause\basicCode{[]}
				\presentationPause\lstinputlisting[numbers=none, linerange={2-2}]{../../code/array/A.sintax.cpp}
				\presentationPause\item Inicialização diferentona
				\presentationPause\lstinputlisting[numbers=none, linerange={2-6}]{../../code/array/A.inicialization.cpp}
			\end{itemize}
		\end{frame}
	
		\begin{frame}{Sequência padrão}
			\begin{itemize}
				\presentationPause\item Acessar os valores por índices entre colchetes \basicCode{[i]}
				\presentationPause\item Começa em $0$
				\presentationPause\item Acaba em $n-1$
			\end{itemize}
			\presentationPause\begin{table}[H]
				\centering
				\caption{Representação de vetor de $n$ valores}
				\label{table.array}
				\begin{tabular}{cccccc}
					$0$                      & $1$                     & $2$                     & $3$                     & $\cdots$                & $n-1$                     \\ \hline
					\multicolumn{1}{|l|}{} & \multicolumn{1}{l|}{} & \multicolumn{1}{l|}{} & \multicolumn{1}{l|}{} & \multicolumn{1}{l|}{} & \multicolumn{1}{l|}{} \\ \hline
					\end{tabular}
				\end{table}
			\presentationPause\lstinputlisting[numbers=none]{../../code/array/A.examples.cpp}
		\end{frame}


	\section{Matrizes}
		\begin{frame}{Sequência de sequências}
			\begin{itemize}
				\presentationPause\item Semelhante ao vetor
				\presentationPause\item Pode ser considerada uma \emph{matriz}
				\presentationPause\item Consistem em um vetor com vetores
				\presentationPause\item Sintaxe de declaração semelhante à do vetor
				\presentationPause\lstinputlisting[numbers=none, linerange={2-2}]{../../code/array/M.sintax.cpp}
				\presentationPause\item Inicialização muito diferentona
				\presentationPause\lstinputlisting[numbers=none, linerange={2-7}]{../../code/array/M.inicialization.cpp}
			\end{itemize}
		\end{frame}

		\begin{frame}{Sequência de sequências}
			\begin{itemize}
				\presentationPause\item Um colchete para cada índice
				\presentationPause\item Também começa em $0$
				\presentationPause\item O passo de vetor para matriz é o passo de matriz para tensor
				\presentationPause\item O \basicCode{for} foi feito pra isso
			\end{itemize}
			\begin{table}[H]
				\centering
				\caption{Representação de matriz de $m\times n$}
				\label{table.matrix}
				\begin{tabular}{ccccccc}
					& $0$                      & $1$                     & $2$                     & $3$                     & $\cdots$                & $n-1$                     \\ \cline{2-7}
					$0$ & \multicolumn{1}{|l|}{} & \multicolumn{1}{l|}{} & \multicolumn{1}{l|}{} & \multicolumn{1}{l|}{} & \multicolumn{1}{l|}{} & \multicolumn{1}{l|}{} \\ \cline{2-7}
					$1$ & \multicolumn{1}{|l|}{} & \multicolumn{1}{l|}{} & \multicolumn{1}{l|}{} & \multicolumn{1}{l|}{} & \multicolumn{1}{l|}{} & \multicolumn{1}{l|}{} \\ \cline{2-7}
					$2$ & \multicolumn{1}{|l|}{} & \multicolumn{1}{l|}{} & \multicolumn{1}{l|}{} & \multicolumn{1}{l|}{} & \multicolumn{1}{l|}{} & \multicolumn{1}{l|}{} \\ \cline{2-7}
					$3$ & \multicolumn{1}{|l|}{} & \multicolumn{1}{l|}{} & \multicolumn{1}{l|}{} & \multicolumn{1}{l|}{} & \multicolumn{1}{l|}{} & \multicolumn{1}{l|}{} \\ \cline{2-7}
					$\vdots$ & \multicolumn{1}{|l|}{} & \multicolumn{1}{l|}{} & \multicolumn{1}{l|}{} & \multicolumn{1}{l|}{} & \multicolumn{1}{l|}{} & \multicolumn{1}{l|}{} \\ \cline{2-7}
					$m-1$ & \multicolumn{1}{|l|}{} & \multicolumn{1}{l|}{} & \multicolumn{1}{l|}{} & \multicolumn{1}{l|}{} & \multicolumn{1}{l|}{} & \multicolumn{1}{l|}{} \\ \cline{2-7}
					\end{tabular}
				\end{table}
			\end{frame}

			\begin{frame}{Sequência de sequências}
				\presentationPause\lstinputlisting[numbers=none]{../../code/array/M.examples.cpp}
			\end{frame}
	

	\section{Strings}
		\begin{frame}{Sequências de caracteres}
			\begin{itemize}
				\presentationPause\item Vetor especial de \basicCode{char}
				\presentationPause\item A forma de inicialização diferentona master
				\presentationPause\lstinputlisting[numbers=none, linerange={2-3}]{../../code/array/S.sintax.cpp}
				\presentationPause\item Tamanho implítico?
				\presentationPause\item Aspas duplas \presentationPause e o caractere nulo (\basicCode{'\0'})
				\presentationPause\item Podemos contar tamanho de \textit{string} com isso?
			\end{itemize}
		\end{frame}

		\begin{frame}{Sequências de caracteres}
			\presentationPause\lstinputlisting[numbers=none]{../../code/array/S.examples.cpp}
		\end{frame}
		
	\section{Preprocessador}
		\begin{frame}{Preprocessador}
			\begin{itemize}
				\presentationPause\item Tamanhos precisam ser literais
				\presentationPause\item E se eu pudesse criar literais com nomes
				\presentationPause\item Agora você pode
				\presentationPause\item \emph{Se ligar agora, 75\% de desconto}
				\presentationPause\item Define pra mim o que você quer... \presentationPause\basicCode{\#define}
			\end{itemize}
			\presentationPause\lstinputlisting[numbers=none]{../../code/preprocessor/define.cpp}
		\end{frame}
	

	\section{Hora de brincar}
		\begin{frame}
			\begin{center}\Huge
				Vamos testar!
			\end{center}
		\end{frame}


\end{document}