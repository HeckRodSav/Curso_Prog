\documentclass[14pt]{beamer}
	
\input{../../template/slideTemplate.txt}
\renewcommand{\DEBUG}{0}
\renewcommand{\PRESENTATION}{0}

\subtitle{Dando mais voltas}
\usecolortheme[named=slideCyan]{structure}

\begin{document}
	
	\begin{frame}
		\titlepage
	\end{frame}

	\begin{frame}
		\tableofcontents
	\end{frame}

	\section{Escolha}
		\begin{frame}{Escolha}
			\begin{itemize}
				\presentationPause\item Qual a diferença entre escolha e decisão?
				\presentationPause\item Problemas da escolha...
				\presentationPause\item Porque então vou usar essa coisa?
			\end{itemize}
			\begin{multicols}{2}
				\presentationPause\lstinputlisting[numbers=none, linerange={2-33}]{../../code/flow/if.sequence.cpp}
			\end{multicols}
		\end{frame}

		\begin{frame}{Não entendeu? Sem problemas}
			\presentationPause
			\begin{figure}[H]
				\centering
				\input{../../flowchart/switch.A.txt}
				\caption{Fluxograma de esturura de decisão composta}
				\label{fig.flow.switch.A}
			\end{figure}
		\end{frame}

		\begin{frame}{O \textit{switch}}
			\begin{itemize}
				\presentationPause\item O que diabos foi aquilo?\presentationPause?\presentationPause?
				\presentationPause\item Como faço pra não ter aquilo?
				\presentationPause\item A estrutura de escolha: \presentationPause o \basicCode{switch}
				\presentationPause\item Recebe o argumento que precisar
				\presentationPause\item Principal desvantagem: \presentationPause é binário\presentationPause(?)
			\end{itemize}
		\end{frame}

		\begin{frame}{O \textit{switch}}
			\begin{multicols}{2}
				\lstinputlisting[numbers=none, linerange={4-24}]{../../code/flow/switch.sintax.cpp}
			\end{multicols}
			Note os detalhes:
			\begin{multicols}{2}
				\begin{itemize}
					\presentationPause\item \basicCode{switch}
					\presentationPause\item \basicCode{case}
					\presentationPause\item \basicCode{break}
					\presentationPause\item \basicCode{default}
				\end{itemize}
			\end{multicols}
		\end{frame}

		% \begin{frame}{Não entendeu? Sem problemas}
		% 	\begin{figure}[H]
		% 		\centering
		% 		\input{../../flowchart/switch.A.txt}
		% 		\caption{Fluxograma de esturura de decisão composta}
		% 		\label{fig.flow.switch.A}
		% 	\end{figure}
		% \end{frame}

		\begin{frame}{Notas novamente}
			\lstinputlisting[linerange={4-15}]{../../code/flow/switch.example.cpp}
		\end{frame}	

	\section{Repete mais}
		\begin{frame}{Repetições definidas}
			\begin{itemize}
				\presentationPause\item Utilizar o \basicCode{while} pra repetição definida?
				\presentationPause\item O \basicCode{while} tem uma estrutura feita para repetição indefinida
				\presentationPause\item Há uma estrutura feita só para estas situação
				\presentationPause\item O fluxograma dela é idêntico ao do \basicCode{while}
			\end{itemize}
			\presentationPause\lstinputlisting[numbers=none, linerange={2-7}]{../../code/flow/while.i.cpp}
		\end{frame}

		\begin{frame}{O \textit{for}}
			% \begin{columns}
			% 	\begin{column}{.65\textwidth}
					\begin{itemize}
						\presentationPause\item Sintaxe simpática
						\presentationPause\item Todo programador gosta
						\presentationPause\item Crucial para sequências de memórias
						\presentationPause\item Recebe três agrupamentos como argumentos\presentationPause(?)
						\presentationPause\item Iteração! \presentationPause $\neq$ interação...
						\presentationPause\item Tem uma variável própria \presentationPause em 95\% das vezes...
						\presentationPause\item \basicCode{++i} $\neq$ \basicCode{i++}
					\end{itemize}
					\presentationPause\lstinputlisting[numbers=none, linerange={2-2}]{../../code/flow/for.line.sintax.cpp}
					\presentationPause\lstinputlisting[numbers=none, linerange={2-7}]{../../code/flow/for.block.sintax.cpp}
				% \end{column}
				% \begin{column}{.35\textwidth}
				% \end{column}
			% \end{columns}
		\end{frame}

		\begin{frame}{Alguns (muitos) detalhes}
			\begin{itemize}
				\presentationPause\item A declaração da variável de controle pode ser feita na região de inicaliação
				\presentationPause\item Mais de uma variável pode ser criada e utilizada (todas do mesmo tipo)
				\presentationPause\item As variáveis de iteração podem ser utilizadas dentro da estrutura de repetição
				\presentationPause\item A condição de continuação não precisa ser atrelada as variáveis criadas
				\presentationPause\item O passo de iteração pode alterar todas as variáveis criadas de forma independente
				\presentationPause\item Não é necessário inicializar a variável de iteração se esta já vem com um valor estabelecido (ou seja, que não é lixo de memória)
				\presentationPause\item O passo de iteração não precisa alterar somente as variáveis de iteração
				\presentationPause\item As variáveis declaradas na regição de inicialização são locais para o comando (ou bloco) da repetição
			\end{itemize}
		\end{frame}

		\begin{frame}
			\lstinputlisting[linerange={2-19}]{../../code/flow/for.example.cpp}
		\end{frame}

	\section{Brutal}
		\begin{frame}{Saídas bruscas}
			\begin{itemize}
				\presentationPause\item Saídas bruscas são úteis
				\presentationPause\item Existe um controle de fluxo para quebra de repetição
				\presentationPause\item Existe um controle de fluxo para pular iteração
				\presentationPause\item Associação a \basicCode{if}
				\presentationPause\item Se não existir \basicCode{if}, existe um problema
				\presentationPause\item Você já viu um deles
			\end{itemize}
			\presentationPause\lstinputlisting[numbers=none,linerange={2-4}]{../../code/flow/continuebreak.sintax.cpp}
		\end{frame}

		\begin{frame}
			\lstinputlisting[linerange={2-14}]{../../code/flow/continuebreak.example.cpp}
		\end{frame}
		
	\section{Hora de brincar}
		\begin{frame}
			\begin{center}\Huge
				Vamos testar!
			\end{center}
		\end{frame}

\end{document}