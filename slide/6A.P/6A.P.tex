\documentclass[14pt]{beamer}
	
\input{../../template/slideTemplate.txt}
\renewcommand{\DEBUG}{0}
\renewcommand{\PRESENTATION}{1}

\subtitle{Não aponte para mim}
\usecolortheme[named=slideYellow]{structure}

\begin{document}
	
	\begin{frame}
		\titlepage
	\end{frame}

	\begin{frame}
		\tableofcontents
	\end{frame}

	\section{Imprimindo vetores}
		\begin{frame}{Joga num \basicCode{cout}}
			\begin{itemize}
				\presentationPause\item Vocês já tentaram imprimir um vetor?
				\presentationPause\item Qual resultado obtiveram?
				\presentationPause\item Vocês tem ideia do que é esse valor?
				\presentationPause\item Esse valor foi sempre igual?
			\end{itemize}
			\presentationPause\lstinputlisting[numbers=none, linerange={2-4}]{../../code/pointers/cout.vetor.cpp}
			\begin{itemize}
				\presentationPause\item Este valor é o endereço onde o vetor está na memória!
			\end{itemize}
		\end{frame}

	\section{Referenciando}
		\begin{frame}{Variáveis também tem endereço}
			\begin{itemize}
				\presentationPause\item Todas as coisas \emph{declaradas} no programa têm um endereço de memória
				\presentationPause\item Este endereço será apresentado como um número hexadecimal de $12$ (?) dígitos
				\presentationPause\item Há um operador criado para coletar este valor
				\presentationPause\item Este operador é chamado \emph{referenciador} \presentationPause ou oprador de endereço
				\presentationPause\item Utiliza o \emph{e comercial} (\basicCode{\&})
				\presentationPause\item Operador unário e de consulta
			\end{itemize}
			\presentationPause\lstinputlisting[numbers=none, linerange={2-2}]{../../code/operators/reference.sintax.cpp}
			\begin{itemize}
				\presentationPause\item E o exemplo? \presentationPause Depois...
				\presentationPause\item Mas podemos exibir valores!
			\end{itemize}
		\end{frame}

	\section{Apontando}
		\begin{frame}{Dedos na cara}
			\begin{itemize}
				\presentationPause\item Legal, pra que vou usar isso?
				\presentationPause\item Com o endereço podemos alterar o valor salvo lá
				\presentationPause\item Para isso utilizamos algo que serve para apontar para estes endereços
				\presentationPause\item Os \emph{ponteiros}
				\presentationPause\item Ponteiros são variáveis especiais expecializada em armazenar endereços
				\presentationPause\item Apresentam os mesmo tipos que as variáveis \presentationPause (inclusive tipos abstratos)
			\end{itemize}
		\end{frame}

		\begin{frame}{Problemas com tamanhos}
			\begin{itemize}
				\presentationPause\item \textit{Espera, você está me dizendo \presentationPause que essas variáveis que sempre armazenam endereços \presentationPause tem mais de um tipo???}
				\presentationPause\item \textit{Mas os endereços não deveriam ter sempre o mesmo formato???}
				\presentationPause\item Sim! \presentationPause e tem um ótimo motivo
				\presentationPause\item Vamos brincar com outro operador rapidamente\dots
				\presentationPause\item O \basicCode{sizeof}, o operador de tamanho
			\end{itemize}
			\presentationPause\lstinputlisting[numbers=none, linerange={2-2}]{../../code/operators/sizeof.sintax.cpp}
			\begin{multicols}{3}
				\begin{itemize}
					\presentationPause\item \textit{O que ele faz?}
					\presentationPause\item \textit{O que ele come?}
					\presentationPause\item \textit{Como se reproduz?}
				\end{itemize}
			\end{multicols}
			\begin{itemize}
				\presentationPause\item Ele retorna a quantidade de bytes que a entidade no argumento ocupa
				\presentationPause\item Vamos fazer um teste\dots
			\end{itemize}
		\end{frame}

		\begin{frame}{Voltando aos ponteiros}
			\begin{itemize}
				\presentationPause\item Legal, com esse operador há uma explicação
				\presentationPause\item Cada tipo tem um tamanho característico (ou tem outras diferenças)
				\presentationPause\item Esse tamanho mostra quantos bytes uma variável representa na memória
				\presentationPause\item Se utilizarmos um ponteiro genérico, não tem como garantir a quantidade correta de bytes alterados no acesso
				\presentationPause\item Legal, faz sentido, mas duas dúvidas:
				\begin{itemize}
					\presentationPause\item Como faz?
					\presentationPause\item Acesso\dots?
				\end{itemize}
				\presentationPause\item Vamos começar com a declaração
			\end{itemize}
		\end{frame}

		\begin{frame}{Declarando}
			\begin{itemize}
				\presentationPause\item A declaração é quase idêntica à de uma variável comum
				\presentationPause\item Ela ganha uma estrelinha! \presentationPause Mentira, é só um asterisco \basicCode{*}
				\presentationPause\item O asterisco está lá só nos ponteiros
			\end{itemize}
			\presentationPause\lstinputlisting[numbers=none, linerange={2-2}]{../../code/pointers/declaration.cpp}
			\begin{itemize}
				\presentationPause\item \textit{Legal, e o tal acesso?}
				\presentationPause\item Precisamos inicializar ainda\dots
			\end{itemize}
		\end{frame}

		\begin{frame}{Inicializando}
			\begin{itemize}
				\presentationPause\item Mesma ideia que uma variável normal
				\presentationPause\item Recebe endereços
				\presentationPause\item Pode receber o valor de outro ponteiro
				\presentationPause\item Pode apontar pra qualquer coisa \presentationPause Mas nem todas são simples acessar\dots
			\end{itemize}
			\presentationPause\lstinputlisting[numbers=none, linerange={2-4}]{../../code/pointers/inicialization.cpp}
		\end{frame}

		\begin{frame}{Uns exemplos}
			\presentationPause\lstinputlisting[numbers=none, linerange={2-9}]{../../code/pointers/examples.1.cpp}
		\end{frame}

	\section{Derreferenciando}
		\begin{frame}{Outro operador}
			\begin{itemize}
				\presentationPause\item Agora chegamos na parte de acesso
				\presentationPause\item Não adianta muito saber o endereço se não souber \textit{mandar cartas}
				\presentationPause\item Existe um operador para acessar através do endereço, utilizando os ponteiros
				\presentationPause\item Este operador é chamado \emph{derreferenciador} \presentationPause ou operador de acesso
				\presentationPause\item Utiliza o asterisco (\basicCode{*})
				\presentationPause\item É um operador unário e de consulta\presentationPause, mas\dots
				\presentationPause\item Ele é de consulta para o valor do ponteiro, mas com ele se altera a o valor apontado
			\end{itemize}
			\presentationPause\lstinputlisting[numbers=none, linerange={2-2}]{../../code/operators/dereference.sintax.cpp}
		\end{frame}

		\begin{frame}{Legal, quando usar?}
			\presentationPause\lstinputlisting[numbers=none]{../../code/pointers/examples.2.cpp}
		\end{frame}

	\section{Ponteiro de membro}
		\begin{frame}{Uma setinha $\rightarrow$}
			\begin{itemize}
				\presentationPause\item Temos um caso especial de ponteiro
				\presentationPause\item Quando apontamos para um tipo abstrato acontece um problema com precedência\dots
				\presentationPause\item A precedência do operador de acesso é menor que a do operador de membro (\basicCode{.})
				\presentationPause\item Podemos utilizar parênteses\dots
			\end{itemize}
			\presentationPause\lstinputlisting[numbers=none]{../../code/operators/dereferenceMemberAccess.non.sintax.cpp}
			\begin{itemize}
				\presentationPause\item Tá ruim\dots
				\presentationPause\item E se existir um operador pra facilitar isso?
				\presentationPause\item Existe! \presentationPause E é uma seta
				\presentationPause\item Aliás, quase uma seta (\basicCode{->})
			\end{itemize}
		\end{frame}

		\begin{frame}{Quase uma seta\dots}
			\presentationPause\lstinputlisting[numbers=none]{../../code/operators/dereferenceMemberAccess.sintax.cpp}
			\begin{itemize}
				\presentationPause\item Utilizamos este operador para acessar membros quando temos apenas o ponteiro
				\presentationPause\item Ele acessa o endereço apontado e, lá, acessa o membro
				\presentationPause\item Muito\presentationPause, muito\presentationPause, muitíssimo\presentationPause, muito mesmo utilizado em esturutras de dados
			\end{itemize}
		\end{frame}

		\begin{frame}{Um caso de uso}
			\presentationPause\lstinputlisting[numbers=none]{../../code/pointers/examples.4.cpp}
		\end{frame}

	\section{Metamatemático \frownie}
		\begin{frame}{Continhas}
			\begin{itemize}
				\presentationPause\item Por que essa carinha triste? \presentationPause\frownie
				\presentationPause\item Podemos ser melhores do que isso\dots
				\presentationPause\item Mas vocês precisam saber que isso axiste
				\presentationPause\item Operadores \basicCode{++} e \basicCode{--} \presentationPause (Soma e subtração também)
				\presentationPause\item Caminhar para frente e para trás
				\presentationPause\item Problema: \presentationPause Perder o valor original
				\presentationPause\item \textit{E daí?}
				\presentationPause\item Vocês entederão isso na próxima aula
				\presentationPause\item Tem a ver com \textbf{\presentationPause A \presentationPause COISA \presentationPause MAIS \presentationPause DIFÍCIL} \presentationPause (que não é difícil)
			\end{itemize}
		\end{frame}

		\begin{frame}{Sente o drama}
			\presentationPause\lstinputlisting[numbers=none]{../../code/pointers/examples.5.cpp}
		\end{frame}

		\begin{frame}{Mais que continhas}
			\begin{itemize}
				\presentationPause\item Indexado!
			\end{itemize}
			\presentationPause\lstinputlisting[numbers=none]{../../code/pointers/examples.3.cpp}
		\end{frame}
		
	\section{Hora de brincar}
		\begin{frame}
			\begin{center}\Huge
				Vamos testar!
			\end{center}
		\end{frame}

	
\end{document}