\documentclass[11pt]{beamer}
	
\input{../../template/slideTemplate.txt}
\renewcommand{\DEBUG}{0}
\renewcommand{\PRESENTATION}{1}

\subtitle{Qual o resultado?}
\usecolortheme[named=slideBlue]{structure}

\begin{document}

\titlepage

\begin{frame}
	\tableofcontents
\end{frame}

\section{Alterações}
	\begin{frame}{Modificando variáveis}
		\begin{itemize}
			\presentationPause\item Variáveis que não variam não são variáveis
			\presentationPause\item As alterações nas variáveis dependem de seu tipo
			\presentationPause\item Existem três grupos de operadores: unários, binários e ternários:
			\begin{itemize}
				\presentationPause\item[Unário] utiliza apenas um valor
				\presentationPause\item[Binário] utiliza dois valores
				\presentationPause\item[Ternário] utiliza três valores
			\end{itemize}
			\presentationPause\item Todo operador retorna o valor de sua operação
		\end{itemize}
	\end{frame}

\section{Unário}
	\begin{frame}{Incrementos e decrementos unitários}
		\begin{itemize}
			\presentationPause\item Variáveis podem ter seu valor acrescido ou descrescido
			\presentationPause\item Estes operadores unitários o fazem com o valor de 1
			\presentationPause\item Podem ser prefixos ou sufixos
			\presentationPause\item Alteram o valor registrado
		\end{itemize}
		\presentationPause\lstinputlisting[numbers=none,linerange={2-6}]{../../code/operators/increment.sintax.cpp}
	\end{frame}\begin{frame}{Incrementos e decrementos unitários}
		\lstinputlisting{../../code/operators/increment.example.cpp}
	\end{frame}

	% \begin{frame}{Sinalizadores}
	% 	\begin{itemize}
	% 		\presentationPause\item Equivalem aos sinais matemáticos
	% 		\presentationPause\item Fazem o mesmo que multiplicar um número por $+1$ ou $-1$
	% 		\presentationPause\item Não alteram o valor registrado
	% 	\end{itemize}
	% 	\presentationPause\lstinputlisting[numbers=none,linerange={2-4}]{../../code/operators/sign.sintax.cpp}
	% \end{frame}\begin{frame}{Sinalizadores}
	% 	\lstinputlisting{../../code/operators/sign.example.cpp}
	% \end{frame}

	\begin{frame}{Negador booleano}
		\begin{itemize}
			\presentationPause\item Retorna o estado inverso de um \basicCode{bool}
			\presentationPause\item Só pode ser usado no tipo \basicCode{bool} \presentationPause(Mas funciona em outros tipos)
			\presentationPause\item Não confundir com fatorial!
			\presentationPause\item Não alteram o valor registrado
		\end{itemize}
		\presentationPause\lstinputlisting[numbers=none,linerange={2-2}]{../../code/operators/neg.sintax.cpp}
	\end{frame}\begin{frame}{Negador booleano}
		\lstinputlisting{../../code/operators/neg.example.cpp}
	\end{frame}

	\begin{frame}{Complemento binário}
			\begin{itemize}
				\presentationPause\item Este operador inverte todos os bits de uma variável
				\presentationPause\item Em tipos \basicCode{signed} o sinal é invertido
				\presentationPause\item Não pode ser utilizado em números flutantes
				\presentationPause\item Não alteram o valor registrado
			\end{itemize}
			\presentationPause\lstinputlisting[numbers=none,linerange={2-2}]{../../code/operators/complement.sintax.cpp}
	\end{frame}\begin{frame}{Complemento binário}
		\lstinputlisting{../../code/operators/complement.example.cpp}
	\end{frame}

\section{Binário}
	% \begin{frame}{Atribuidor simples}
	% 	\begin{itemize}
	% 		\presentationPause\item O operador mais utilizado é o atribuidor simples
	% 		\presentationPause\item Ele atribui o valor a direita à variável a direita
	% 		\presentationPause\item Cuidado para não inverter
	% 		\presentationPause\item Não atribuir tipos a variáveis de outros tipos
	% 	\end{itemize}
	% 	\presentationPause\lstinputlisting[numbers=none,linerange={2-2}]{../../code/operators/assignment.sintax.cpp}
	% \end{frame}\begin{frame}{Atribuidor simples}
	% 	\lstinputlisting{../../code/operators/assignment.example.cpp}
	% \end{frame}

	% \begin{frame}{Aritméticos}
	% 	\begin{itemize}
	% 		\presentationPause\item As quatro operações básicas da matemática
	% 		\presentationPause\item Os símbolos padrão
	% 		\presentationPause\item Não alteram o valor registrado
	% 		\presentationPause\item Cuidado para não dividir por 0
	% 	\end{itemize}
	% 	\presentationPause\lstinputlisting[numbers=none,linerange={2-6}]
	% 	{../../code/operators/arithmetic.sintax.cpp}
	% 	\presentationPause\begin{equation}\nonumber
	% 		\begin{array}{cc}
	% 			\text{\basicCode{E}} & \multicolumn{1}{|c}{\text{\basicCode{F}}} \\ \cline{2-2} 
	% 			\text{\basicCode{G}} & \text{\basicCode{H}}
	% 		\end{array} \Rightarrow \begin{array}{cc}
	% 			13 & \multicolumn{1}{|c}{5} \\ \cline{2-2} 
	% 				3 & 2		
	% 		\end{array}
	% 	\end{equation}
	% \end{frame}\begin{frame}{Aritméticos}
	% 	\lstinputlisting{../../code/operators/arithmetic.example.cpp}
	% \end{frame}

	\begin{frame}{Deslocadores}
		\begin{itemize}
			\presentationPause\item Os deslocadores são operações bit-a-bit
			\presentationPause\item São operadores que multiplicam o valor registrado por potências de 2
			\presentationPause\item Não alteram o valor registrado
			\presentationPause\item É mais rápido do que uma multiplicação comum
		\end{itemize}
		\presentationPause\lstinputlisting[numbers=none,linerange={2-3}]{../../code/operators/shift.sintax.cpp}
		\begin{equation}\nonumber
		V\!\cdot\!2^{+S} \ \ \ \ \ \ \ \ \ \ V\!\cdot\!2^{-S}
		\end{equation}
	\end{frame}\begin{frame}{Deslocadores}
			\lstinputlisting{../../code/operators/shift.example.cpp}
	\end{frame}

	\begin{frame}{Lógicos bit-a-bit}
		\begin{itemize}
			\presentationPause\item Operadores lógicos normais
			\presentationPause\item Trabalham separadamente a cada bit
		\end{itemize}
		\presentationPause\lstinputlisting[numbers=none,linerange={2-4}]{../../code/operators/bitwise.sintax.cpp}
		\presentationPause\begin{table}[!h]
			\centering
			\caption{Tabela verdade para operadores lógicos}
			\label{table.truth}
			\begin{tabular}{c|c||c|c|c|c|c}
				A & B & $_\text{\emph{NOT}}$ A & $_\text{\emph{NOT}}$ B & A $_\text{\emph{OR} }$B & A $_\text{\emph{AND}}$ B & A $_\text{\emph{XOR}}$ B \\\hline
				0 & 0 & 1 & 1 & 0 & 0 & 0 \\
				0 & 1 & 1 & 0 & 1 & 0 & 1 \\
				1 & 0 & 0 & 1 & 1 & 0 & 1 \\
				1 & 1 & 0 & 0 & 1 & 1 & 0 \\
			\end{tabular}
		\end{table}
	\end{frame}\begin{frame}{Lógicos bit-a-bit}
		\lstinputlisting{../../code/operators/bitwise.example.cpp}
		\presentationPause\begin{equation}\nonumber
			\begin{array}{cc}
				&	10101011\\
				_\text{\emph{OR}}\!	&	01100100\\\cline{2-2}
				&	11101111
			\end{array}\  
			\begin{array}{cc}
				&	10101011\\
				_\text{\emph{AND}}\!	&	01100100\\\cline{2-2}
				&	00100000
			\end{array}\
			\begin{array}{cc}
				&	10101011\\
				_\text{\emph{XOR}}\! &	01100100\\\cline{2-2}
				&	11001111
			\end{array}
		\end{equation}
	\end{frame}

	\begin{frame}{Atribuidor composto}
		\begin{itemize}
			\presentationPause\item Uma simplificação de operações que atribuem valores
			\presentationPause\item Operações mais simples são sempre escritas desta maneira
		\end{itemize}
		\presentationPause\lstinputlisting[numbers=none,linerange={2-4}]{../../code/operators/compound.sintax.cpp}
	\end{frame}\begin{frame}{Atribuidor composto}
		\begin{table}[!h]
			\small
			\centering
			\caption{Relação de operadores de atribuição composta e seus equivalentes}
			\label{table.compound}
			\begin{tabular}{rclcrcccl}
				\multicolumn{3}{c}{Composto}	&  & \multicolumn{5}{c}{Equivalente}\\\hline
				\basicCode{A} & \basicCode{+=} & \basicCode{B;} & $\Leftrightarrow$ & \basicCode{A} & \basicCode{=} & \basicCode{A} & \basicCode{+} & \basicCode{B;} \\
				\basicCode{A} & \basicCode{-=} & \basicCode{B;} & $\Leftrightarrow$ & \basicCode{A} & \basicCode{=} & \basicCode{A} & \basicCode{-} & \basicCode{B;} \\
				\basicCode{A} & \basicCode{*=} & \basicCode{B;} & $\Leftrightarrow$ & \basicCode{A} & \basicCode{=} & \basicCode{A} & \basicCode{*} & \basicCode{B;} \\
				\basicCode{A} & \basicCode{/=} & \basicCode{B;} & $\Leftrightarrow$ & \basicCode{A} & \basicCode{=} & \basicCode{A} & \basicCode{/} & \basicCode{B;} \\
				\basicCode{A} & \basicCode{\%=} & \basicCode{B;} & $\Leftrightarrow$ & \basicCode{A} & \basicCode{=} & \basicCode{A} & \basicCode{\%} & \basicCode{B;} \\
				\basicCode{A} & \basicCode{>>=} & \basicCode{B;} & $\Leftrightarrow$ & \basicCode{A} & \basicCode{=} & \basicCode{A} & \basicCode{>>} & \basicCode{B;} \\
				\basicCode{A} & \basicCode{<<=} & \basicCode{B;} & $\Leftrightarrow$ & \basicCode{A} & \basicCode{=} & \basicCode{A} & \basicCode{<<} & \basicCode{B;} \\
				\basicCode{A} & \basicCode{\|=} & \basicCode{B;} & $\Leftrightarrow$ & \basicCode{A} & \basicCode{=} & \basicCode{A} & \basicCode{\|} & \basicCode{B;} \\
				\basicCode{A} & \basicCode{\&=} & \basicCode{B;} & $\Leftrightarrow$ & \basicCode{A} & \basicCode{=} & \basicCode{A} & \basicCode{\&} & \basicCode{B;} \\
				\basicCode{A} & \basicCode{^=} & \basicCode{B;} & $\Leftrightarrow$ & \basicCode{A} & \basicCode{=} & \basicCode{A} & \basicCode{^} & \basicCode{B;}
			\end{tabular}
		\end{table}
	\end{frame}\begin{frame}{Atribuidor composto}
		\lstinputlisting{../../code/operators/compound.example.cpp}
	\end{frame}

	% \begin{frame}{Comparadores}
	% 	\begin{itemize}
	% 		\presentationPause\item Verificar se valores são iguais ou diferentes
	% 		\presentationPause\item Descobrir se um valor é maior que outro
	% 	\end{itemize}
	% 	\presentationPause\lstinputlisting[numbers=none,linerange={2-7}]{../../code/operators/comparison.sintax.cpp}
	% \end{frame}\begin{frame}{Comparadores}
	% 	\lstinputlisting{../../code/operators/comparison.example.cpp}
	% \end{frame}

	\begin{frame}{Lógicos Booleanos}
		\begin{itemize}
			\presentationPause\item Servem para fazer junção de tipos \basicCode{bool}
			\presentationPause\item Montar expressões de dependências lógicas mais compostas
		\end{itemize}
		\presentationPause\lstinputlisting[numbers=none,linerange={2-3}]{../../code/operators/logic.sintax.cpp}
	\end{frame}\begin{frame}{Lógicos Booleanos}
		\lstinputlisting{../../code/operators/logic.example.cpp}
	\end{frame}

\section{Ternário}
	\begin{frame}{Operador ternário}
		\begin{itemize}
			\presentationPause\item Não tem nome próprio \presentationPause\frownie
			\presentationPause\item Faz escolhas a partir de decisões
			\presentationPause\item Não altera o fluxo do código
			\presentationPause\item É simpático
		\end{itemize}
		\presentationPause\lstinputlisting[numbers=none,linerange={2-2}]{../../code/operators/ternary.sintax.cpp}
	\end{frame}\begin{frame}{Operador ternário}
		\lstinputlisting{../../code/operators/ternary.example.cpp}
	\end{frame}

\section{Precedência}
	\begin{frame}{Precedência}
		\begin{itemize}
			\presentationPause\item Os operadores tem preferência de ordem
			\presentationPause\item \basicCode{1 + 1 + 1 + 1 * 0 = } ?
		\end{itemize}
		\presentationPause\lstinputlisting[numbers=none,linerange={2-2}]{../../code/operators/precedence.example.1.cpp}
		\presentationPause\lstinputlisting[numbers=none,linerange={2-2}]{../../code/operators/precedence.example.2.cpp}
	\end{frame}\begin{frame}{Precedência}
		\begin{itemize}
			\presentationPause\item Ambiguidades
			\presentationPause\item Mudando a precedência
			\presentationPause\item Operador de preferência
			\presentationPause\item Igual a matemática
		\end{itemize}
	\end{frame}\begin{frame}{Precedência}
		\begin{table}[!h]
			\tiny\centering
			\caption{Ordem de precedência de operadores}
		
			\label{table.precedence}
			\begin{tabular}{cc}
				Operador & Descrição \\\hline
				\basicCode{()} & preferencal\\\hline\presentationPause
				\basicCode{++}, \basicCode{--} & posfixo \\\hline\presentationPause
				\basicCode{++}, \basicCode{--} & prefixo \\\hline\presentationPause
				\basicCode{\~}, \basicCode{!} & lógico \\\hline\presentationPause
				\basicCode{+}, \basicCode{-} & sinalizadore \\\hline\presentationPause
				\basicCode{*}, \basicCode{/}, \basicCode{\%} & aritimético \\
				\basicCode{+}, \basicCode{-} & aritimético \\\hline\presentationPause
				\basicCode{<<}, \basicCode{>>} & deslocador \\\hline\presentationPause
				\basicCode{<}, \basicCode{<=}, \basicCode{>=}, \basicCode{>} & comparador \\
				\basicCode{==}, \basicCode{!=} & comparador \\\hline\presentationPause
				\basicCode{\&} & lógico \\
				\basicCode{\^} & lógico \\
				\basicCode{|} & lógico \\
				\basicCode{\&\&} & lógico \\
				\basicCode{\|\|} & lógico \\\hline\presentationPause
				\basicCode{=}, \basicCode{+=}, \basicCode{-=}, \basicCode{*=}, \basicCode{/=}, \basicCode{\%=}, \basicCode{\&=}, \basicCode{\^=}, \basicCode{\|=}, \basicCode{<<=},  \basicCode{>>=} & atribuidor \\\hline\presentationPause
				\basicCode{?:} & ternário
			\end{tabular}
		\end{table}
	\end{frame}\begin{frame}{Precedência}
		\lstinputlisting{../../code/operators/precedence.example.cpp}
	\end{frame}

\section{Hora de brincar}
	\begin{frame}
		\begin{center}\Huge
			Vamos testar!
		\end{center}
	\end{frame}

\end{document}