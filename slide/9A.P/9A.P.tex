\documentclass[14pt]{beamer}
	
\input{../../template/slideTemplate.txt}
\renewcommand{\DEBUG}{0}
\renewcommand{\PRESENTATION}{1}

\subtitle{Plante árvores}
\usecolortheme[named=slideOrange]{structure}

\begin{document}
	
	\begin{frame}
		\titlepage
	\end{frame}

	\begin{frame}
		\tableofcontents
	\end{frame}

	\section{O que são estruturas de dados}
		\begin{frame}{Estruturas de dados}
			\begin{itemize}
				\presentationPause\item Armazenar quantidade indefinida
				\presentationPause\item Localizar
				\presentationPause\item Ordenar
				\presentationPause\item Qualquer tipo de item pode ser armazenado
				\presentationPause\item É possível criar várias versões de estruturas de dados
				\presentationPause\item Podem ter vários tipos de regras
				\presentationPause\item São muito versáteis
			\end{itemize}
		\end{frame}

		\begin{frame}{Células}
			\presentationPause Antes de começar, precisamos conhecer o formato de uma célula:

			\begin{columns}
				\begin{column}{.6\textwidth}
					\begin{itemize}
						\presentationPause\item Deve ser capaz de armazenar elemento \presentationPause\textit{ou elementos}
						\presentationPause\item Deve ser capaz de conectar-se a outra célula \presentationPause\textit{ou células}
						\presentationPause\item Normalmente é armazenada na \textit{heap}
						\presentationPause\item É recomendada a presença de uma célula sentinela \presentationPause (não será eliminado)
						\presentationPause\item Da forma que analizaremos:
						\begin{itemize}
							\presentationPause\item Elemento único
							\presentationPause\item Conecta-se a duas células
							\presentationPause\item Armazenado na \textit{heap}
						\end{itemize}
					\end{itemize}
				\end{column}
				\begin{column}{.35\textwidth}
					\presentationPause\lstinputlisting[numbers=none]{../../code/dataStructure/cell.cpp}
				\end{column}
			\end{columns}
		\end{frame}

		\begin{frame}{Célula}
			\begin{figure}[H]
				\centering
				\input{../../flowchart/cell.txt}
				% \caption{Célula}
				\label{fig.cell}
			\end{figure}
		\end{frame}

	\section{Estruturas lineares}
		\begin{frame}{Um depois do outro}
			\begin{itemize}
				\presentationPause\item Podemos colocar essas células em sequência
				\presentationPause\item Elas ficam ligadas umas às outras \presentationPause \emph{encadeada}
				\presentationPause\item A forma que elas são ligadas pode determinar a estrutura de dados
				\begin{itemize}
					\presentationPause\item Simplesmente encadeada
					\presentationPause\item Duplamente encadeada
					\presentationPause\item Árvore
				\end{itemize}
				\presentationPause\item A forma que os dados são acessados determina a estrutura de dados
				\begin{itemize}
					\presentationPause\item Fila
					\presentationPause\item Pilha
					\presentationPause\item Lista
					\presentationPause\item Árvore
				\end{itemize}
			\end{itemize}
		\end{frame}

		\begin{frame}{A linha}
			\begin{itemize}
				\presentationPause\item A diferença entre os encadeamentos é que um é simples e outro duplo
				\presentationPause\item \textit{Dã?}
				\presentationPause\item Em outras palavas\dots \presentationPause Um só vai, outro vai e volta
			\end{itemize}
			\presentationPause\begin{figure}[H]
				\centering
				\input{../../flowchart/dataStructure.open.txt}
				% \caption{Célula}
				\label{fig.dataStructure.open}
			\end{figure}
			\begin{itemize}
				\presentationPause\item Olhar para só em cima ou só em baixo, é simplesmente encadeada
				\presentationPause\item Olhar a estrutura completa, é duplamente encadeada?
				\presentationPause\item \textit{Mas\dots As pontas estão abertas\dots}
			\end{itemize}
		\end{frame}

		\begin{frame}{A linha?}
			\begin{figure}[H]
				\centering
				\input{../../flowchart/dataStructure.close.txt}
				% \caption{Célula}
				\label{fig.dataStructure.close}
			\end{figure}
		\end{frame}

		\begin{frame}{Detalhes de processamento}
			\begin{itemize}
				\presentationPause\item Deve existir um ao menos ponteiro para uma célula da estrutura
				\presentationPause\item É útil fazer duplamente encadeado, mas mais trabalhoso
				\presentationPause\item Tudo é trabalhado com ponteiros
				\presentationPause\item É prático utilizar recursão\presentationPause, mas cuidado com o encadeamento
				\presentationPause\item É mais seguro utilizar estrutura de repetição\dots
				\presentationPause\item Vale a pena criar uma classe para gerenciar as estruturas
				\begin{itemize}
					\presentationPause\item \basicCode{insere} e \basicCode{remove}?
					\presentationPause\item \basicCode{insere}, \basicCode{desempilha} e \basicCode{desenfila}?
				\end{itemize}
			\end{itemize}
		\end{frame}

		\begin{frame}{Filas e pilhas (e listas)}
			\begin{itemize}
				\presentationPause\item São estruturas irmãs
				\presentationPause\item O processamento é extremamente semelhante
				\presentationPause\item Grande diferença: \presentationPause \basicCode{FIFO} e \basicCode{LIFO}
				\presentationPause\item Filas são \basicCode{FIFO}, enquanto pilhas são \basicCode{LIFO}
				\presentationPause\item \basicCode{FIFO}: \basicCode{First In, First Out}
				\presentationPause\item \basicCode{LIFO}: \basicCode{Last In, First Out}
				\presentationPause\item Ou seja, a diferença é qual item removemos
				\presentationPause\item \textit{E as listas?}
				\presentationPause\item A característica da lista é acessar por índice
				\presentationPause\item Além escolher onde inserir e onde remover
			\end{itemize}
		\end{frame}

	\section{Árvores}
		\begin{frame}{Algumas folhas}
			\begin{itemize}
				\presentationPause\item Cada célula é ligada a duas diferentes
				\presentationPause\item Nenuma célula é ligada a outra
				\presentationPause\item Não é possível existir cíclos
				\presentationPause\item O processamento, normalmente, é recursivo
				\presentationPause\item Amplamente utilizada na computação
				\presentationPause\item \emph{Infelizmente eu não consegui desenhar\dots}
			\end{itemize}
		\end{frame}

		\begin{frame}{A árvore}
			\begin{figure}[H]
				\centering
				\input{../../flowchart/tree.txt}
				% \caption{Célula}
				\label{fig.tree}
			\end{figure}
		\end{frame}

	\section{Hora de brincar}
		\begin{frame}
			\begin{center}\Huge
				Vamos testar!
			\end{center}
		\end{frame}
	
\end{document}