\documentclass[14pt]{beamer}
	
\input{../../template/slideTemplate.txt}
\renewcommand{\DEBUG}{0}
\renewcommand{\PRESENTATION}{1}

\subtitle{O primeiro tipo do bebê}
\usecolortheme[named=slideGreen]{structure}

\begin{document}
	
	\begin{frame}
		\titlepage
	\end{frame}

	\begin{frame}
		\tableofcontents
	\end{frame}

	\section{Enumeradores}
		\begin{frame}{Listando opções}
			\begin{itemize}
				\presentationPause\item Qual previsão do tempo pra hoje?
				\begin{itemize}
					\presentationPause\item Chuva?
					\presentationPause\item Sol?
					\presentationPause\item Nublado?
					\presentationPause\item Neve?
					\presentationPause\item Tufão?
					\presentationPause\item Tsunami?
					\presentationPause\item Chuva \presentationPause de meteoros?
					\presentationPause\item Lagarto gigante?
				\end{itemize}
				\presentationPause\item Montar um programa que mostra o que fazer conforme previsão do tempo
				\presentationPause\item Mas como fazer a escolha? \presentationPause Talvez um código para cada clima\dots
				\begin{enumerate}
					\presentationPause\item Chuva \presentationPause$\rightarrow$ pegar guarda-chuva
					\presentationPause\item Sol \presentationPause$\rightarrow$ pegar protetor solar
					\presentationPause\item Nublado \presentationPause$\rightarrow$ pegar um casaco
					\presentationPause\item Neve \presentationPause$\rightarrow$ pegar um chocolate quente
					\presentationPause\item Tufão \presentationPause$\rightarrow$ fugir para o porão \presentationPause(?)
					\presentationPause\item Tsunami \presentationPause$\rightarrow$ fugir para as montanhas \presentationPause(?)
					\presentationPause\item Chuva de meteoros \presentationPause$\rightarrow$ fugir pra lua (!)
					\presentationPause\item Lagarto gigante \presentationPause$\rightarrow$ chamar os King Kong (!)
				\end{enumerate}
			\end{itemize}
		\end{frame}

		\begin{frame}{Legal, e?}
			\begin{itemize}
				\presentationPause\item Ajudaria muito usar um \basicCode{switch}
				\presentationPause\item Qual código atribuir para cada situação?
				\begin{itemize}
					\presentationPause\item Ordem lexicográfica? \presentationPause(alfabética)
					\begin{itemize}
						\presentationPause\item Tufão, tornado, furacão, ciclone\dots
					\end{itemize}
					\presentationPause\item Nível de periculosidade?
					\begin{itemize}
						\presentationPause\item Um meteoro pode gerar uma tsunami\dots
					\end{itemize}
					\presentationPause\item Tamanho da palavra?
				\end{itemize}
				\presentationPause\item Basicamente precisamos enumerar estes itens
				\presentationPause\item E se existisse um modo de colocar códigos de maneira mais simples?
				\presentationPause\item E se exitisse um tipo que só pode receber valores pré determinados?
				\presentationPause\item E se este tipo só pudesse receber um valor entre os descritos?
				\presentationPause\item Tudo isso é possivel \presentationPause com o poder da \textbf{{\color{slideBlue}i}{\color{slideCyan}m}{\color{slideTurquoise}a}{\color{slideGreen}g}{\color{slideYellow}i}{\color{slideOrange}n}{\color{slideRed}a}{\color{slidePurple}ç}{\color{slideBlue}ã}{\color{slideCyan}o}}
				\presentationPause\item É o que?
			\end{itemize}
		\end{frame}

		\begin{frame}{Ilustrando}
			\begin{itemize}
				\presentationPause\item Primeiro vamos ver o código sem estas vantagens
				\presentationPause\item Lembrando que o que define comando é o ponto-e-vírgula
			\end{itemize}
			\presentationPause\lstinputlisting[numbers=none]{../../code/types/non.enum.example.cpp}
			\begin{itemize}
				\presentationPause\item Parece O.K.
				\presentationPause\item Mas e se a variável não estiver dentro dos valores válidos?
			\end{itemize}
		\end{frame}

		\begin{frame}{Enumerando}
			\begin{itemize}
				\presentationPause\item Podemos definir um novo tipo
				\begin{itemize}
					\presentationPause\item Apenas recebe valores específicos
					\presentationPause\item Apenas recebe valores por palavras-chave
					\presentationPause\item O programador escolhe as palavras-chave
					\presentationPause\item Este tipo precisa ter um nome
				\end{itemize}
				\presentationPause\item A palavra-chave \basicCode{enum} é utilizada para criar estes novos tipos
				\presentationPause\item Operadores são inconsistentes \presentationPause mas podemos mudar isso utilizando polimorfismo
				\presentationPause\item A sintaxe lembra vetores
			\end{itemize}
			\presentationPause\lstinputlisting[numbers=none, linerange={2-5}]{../../code/types/enum.sintax.cpp}
		\end{frame}

		\begin{frame}{Voltando ao clima}
			\presentationPause\lstinputlisting[numbers=none]{../../code/types/enum.example.cpp}
		\end{frame}
		
	\section{Tipos Abstrados de Dados}
		\begin{frame}{Abstraia}
			\begin{itemize}
				\presentationPause\item Pensa numa construção
				\presentationPause\item Ela pode ter portas\presentationPause, quantas?
				\presentationPause\item Ela pode ter janelas\presentationPause, quantas?
				\presentationPause\item Ela pode ter cômodos\presentationPause, quantos?
				\presentationPause\item Ela pode andares\presentationPause, quantos?
				\presentationPause\item A faichada é pintada de alguma cor\presentationPause, qual?
				\presentationPause\item Podemos criar variáveis para armazenar cada valor
				\presentationPause\item Mas isso não garante que estes valores estarão associados
				\presentationPause\item E se existir outra construção? \presentationPause Mais variáveis?
			\end{itemize}
		\end{frame}

	\section{Estruturas}
		\begin{frame}{Estruture}
			\begin{itemize}
				\presentationPause\item É possível criar um tipo que armazene mais de um valor? \presentationPause Vetores!
				\presentationPause\item Mas vetores só funcionam para valores do mesmo tipo
				\presentationPause\item Será que existe algo que armazene vários valores de tipos diferentes?
				\presentationPause\item Sim! \presentationPause A \basicCode{struct}
				\presentationPause\item Semelhante ao \basicCode{union}
				\presentationPause\item Dentro dela, colocamos várias variáveis
				\presentationPause\item E o mais legal: \presentationPause ela é um tipo, e podemos declarar variáveis dela
			\end{itemize}
			\presentationPause\lstinputlisting[numbers=none, linerange={2-11}]{../../code/types/struct.sintax.cpp}
		\end{frame}

		\begin{frame}{Horário}
			\presentationPause\lstinputlisting[numbers=none]{../../code/types/struct.example.cpp}
		\end{frame}
		
	\section{Uniões}
		\begin{frame}{Unidos somos um}
			\begin{itemize}
				\presentationPause\item Irmão estranho da \basicCode{struct}
				\presentationPause\item Só uma das variáveis pode ser utilizada
				\presentationPause\item Como assim mano?
				\presentationPause\item Ele só reserva memória pra salvar um dos valores
				\presentationPause\item A memória reservada é a com o tamanho da maior variável
				\presentationPause\item Usado para casos de valor exclusivo
			\end{itemize}
			\presentationPause\lstinputlisting[numbers=none, linerange={2-11}]{../../code/types/union.sintax.cpp}
		\end{frame}

		\begin{frame}{Mostra ae}
			\presentationPause\lstinputlisting[numbers=none]{../../code/types/union.example.cpp}
		\end{frame}

	\section{Hora de brincar}
		\begin{frame}
			\begin{center}\Huge
				Vamos testar!
			\end{center}
		\end{frame}

	
\end{document}