\documentclass[14pt]{beamer}
	
\input{../../template/slideTemplate.txt}
\renewcommand{\DEBUG}{0}
\renewcommand{\PRESENTATION}{0}

\subtitle{Sente, deite, role}
\usecolortheme[named=slideOrange]{structure}

\begin{document}
	
	\begin{frame}
		\titlepage
	\end{frame}

	\begin{frame}
		\tableofcontents
	\end{frame}

	\section{Pensando em objetos}
		\begin{frame}
			\begin{itemize}
				\presentationPause\item Praticamente uma \basicCode{struct}
				\presentationPause\item Definimos \emph{classes}
				\presentationPause\item Declaramos \emph{objetos}
				\presentationPause\item Objetos são como variáveis
				\presentationPause\item Assim como os carros são como as lanchas
				\presentationPause\item As motos como os \textit{jet skis}
				\presentationPause\item Os pedestres são como os banhistas
			\end{itemize}
		\end{frame}

	\section{Pedestres}
		\begin{frame}{Atributos}
			\begin{itemize}
				\presentationPause\item Aqui suas variáveis tem outro nome
				\presentationPause\item Cada instância define novos atributos
				\presentationPause\item 
			\end{itemize}
		\end{frame}

	\section{Carros}
		\begin{frame}{Métodos}
			\begin{itemize}
				\presentationPause\item Aqui as funções tem outro nome
			\end{itemize}
		\end{frame}

	\section{Emcapsulamento}
		\begin{frame}{Só meu}
			
		\end{frame}

	\section{Hora de brincar}
		\begin{frame}
			\begin{center}\Huge
				Vamos testar!
			\end{center}
		\end{frame}

	
\end{document}