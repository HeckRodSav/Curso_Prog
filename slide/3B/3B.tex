\documentclass[14pt]{beamer}
	
\input{../../template/slideTemplate.txt}
\renewcommand{\DEBUG}{0}
\renewcommand{\PRESENTATION}{0}

\subtitle{Funcional}
\usecolortheme[named=slideCyan]{structure}

\begin{document}
	
	\begin{frame}
		\titlepage
	\end{frame}

	\begin{frame}
		\tableofcontents
	\end{frame}

	\section{I/O \textit{stream}}
	\begin{frame}{Escrita - \textit{O}}
		\begin{itemize}
			\presentationPause\item Já sabemos exibir valores na tela
			\presentationPause\item Tem uma forma muito mais fácil\dots
			\presentationPause\item Você não precisa nem saber o que vai mostrar
			\presentationPause\item Mas\dots  \presentationPause Não é bom pra formatar a saída
			\presentationPause\item \basicCode{cout}
		\end{itemize}
		\presentationPause\lstinputlisting{../../code/minimal/basic.example.HW.cout.cpp}
	\end{frame}

	\begin{frame}{Leitura - \textit{I}}
		\begin{itemize}
			\presentationPause\item E existem alguma coisa que faça leitura tão fácil assim?
			\presentationPause\item É amigo do programador
			\presentationPause\item Você só precisa passar a variável
			\presentationPause\item Mas\dots  \presentationPause A entrada depende do tipo da variável passada
			\presentationPause\item \basicCode{cin}  
		\end{itemize}
		\presentationPause\lstinputlisting{../../code/minimal/basic.example.HW.cin.cpp}
	\end{frame}

	\begin{frame}{Por quê?}
		\begin{itemize}
			\presentationPause\item Por que você me fez sofrer até agora?
			\presentationPause\item Não é porque eu queira vez sua angústia
			\presentationPause\item É necessário conhecer a base
			\presentationPause\item \basicCode{iostream} é avançado e trabalha com conceitos que não vimos
		\end{itemize}
	\end{frame}

	\section{Procedimentos}
		\begin{frame}{Procedural}
			\begin{itemize}
				\presentationPause\item Lembra do tipo \basicCode{void} que você esnobava quando criança?
				\presentationPause\item Ele cresceu\dots
				\presentationPause\item Procedimentos são blocos de comandos, nomeados
				\presentationPause\item Podem ser invocados em qualquer lugar do programa
				\presentationPause\item Melhor que copiar sequências
				\presentationPause\item Deixa o código mais legível
				\presentationPause\item Facilita na hora de achar erros
				\presentationPause\item Podem trabalhar com variáveis globais
				\presentationPause\item Exige um bloco\dots
			\end{itemize}
			\presentationPause\lstinputlisting{../../code/functions/foo.void.sintax.cpp}
		\end{frame}

		\begin{frame}{Euler}
			\presentationPause\lstinputlisting{../../code/functions/foo.void.example.cpp}
		\end{frame}

	\section{Argumentos}
		\begin{frame}{Paramétrico}
			\begin{itemize}
				\presentationPause\item Nem sempre ler variável global é útil
				\presentationPause\item Nem sempre existir variável gobal é útil
				\presentationPause\item Parametros entram como argumentos na forma de variáveis
				\presentationPause\item Repeite a ordem
				\presentationPause\item Separação por vírgula
				\presentationPause\item Ainda devolve o valor por variável global
			\end{itemize}
			\presentationPause\lstinputlisting{../../code/functions/foo.void.args.sintax.cpp}
		\end{frame}

		\begin{frame}{Euler}
			\presentationPause\lstinputlisting{../../code/functions/foo.void.args.example.cpp}
		\end{frame}


	\section{Funções}
		\begin{frame}{Funcinal}
			\begin{itemize}
				\presentationPause\item Devolver valor calculado
				\presentationPause\item Pode ser invocadas em qualquer lugar do programa
				\presentationPause\item Leva o tipo do retorno
				\presentationPause\item Palavra-chave: \presentationPause o \basicCode{return}
				\presentationPause\item Retorno é obrigatório
				\presentationPause\item Porque o \basicCode{main} tem tipo \basicCode{int} e retorna \basicCode{0}
			\end{itemize}
			\presentationPause\lstinputlisting{../../code/functions/foo.type.args.sintax.cpp}
		\end{frame}

		\begin{frame}{Euler}
			\presentationPause\lstinputlisting{../../code/functions/foo.type.args.example.cpp}
		\end{frame}


	\section{Hora de brincar}
		\begin{frame}
			\begin{center}\Huge
				Vamos testar!
			\end{center}
		\end{frame}


\end{document}